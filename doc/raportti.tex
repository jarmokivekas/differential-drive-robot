\documentclass[]{article} % A4 paper and 11pt font size

\usepackage{listings} % Required for inserting code snippets
\usepackage[usenames,dvipsnames]{color} % Required for specifying custom colors and referring 

\usepackage[utf8]{inputenc} % Use 8-bit encoding that has 256 glyphs
\usepackage[finnish]{babel} % language/hyphenation
\usepackage{amsmath,amsfonts,amsthm,mathtools} % Math packages
\usepackage{graphicx}
\usepackage{setspace}
\usepackage[margin=1.0in]{geometry}
%\newcommand{\sinc}{\text{sinc}}


\newcommand{\sinc}{\text{sinc}}

\setlength\parindent{0pt}
\setlength{\parskip}{2mm plus1mm minus1mm}



\usepackage{sectsty} % Allows customizing section commands
%\allsectionsfont{\normalfont\scshape} % Make all sections centered, the default font and small caps

\usepackage{fancyhdr} % Custom headers and footers
%\renewcommand{\familydefault}{\sfdefault}

\pagestyle{fancyplain} % Makes all pages in the document conform to the custom headers and footers
\fancyhead{} % No page header - if you want one, create it in the same way as the footers below
\fancyfoot[L]{Matlab-harjoitustyö 2015} % Empty left footer
\fancyfoot[C]{} % Empty center footer
\fancyfoot[R]{\thepage} % Page numbering for right footer
\renewcommand{\headrulewidth}{0pt} % Remove header underlines
\renewcommand{\footrulewidth}{0pt} % Remove footer underlines
\setlength{\headheight}{13.6pt} % Customize the height of the header
\numberwithin{equation}{section}
\numberwithin{figure}{section}
\numberwithin{table}{section}

%\usepackage[margin=1.5in]{geometry}
%\addtolength{\topmargin}{-.875in}
%\addtolength{\textheight}{1.75in}
%\setlength\parindent{0pt} % Removes all indentation from paragraphs - comment this line for an assignment with lots of text
% \tolerance=1200


\newcommand{\horrule}[1]{\rule{\linewidth}{#1}} % Create horizontal rule command with 1 argument of height

\title{	
\normalfont \normalsize 
\textsc{Turun ammattikorkeakoulu} \\ [25pt] % Your university, school and/or department name(s)
\huge Matlab-harjoitustyöt 2015, Matematiikka 4\\ % The assignment title
}
\author{
Jarmo Kivekäs 1302928
}
\date{\normalsize\today} % Today's date or a custom date



\begin{document}
\maketitle
\tableofcontents

\newpage












\section{Johdanto}
\label{sec:Johdanto}


Tämä raportti käsittelee robottia joka toteutettiin harjoitustyönä Ohjelmoinnin perusteet kurssia varten.
Harjoitustyön tarkoituksen oli oppia järjestelmäläheistä ohjelmointia ARM sekä AVR ympäristäissä.
Raportin sekä siin esitetyn materiaalin on laatinut Jarmo Kivekäs.


\section{Järjestelmä arkkitehtuuri}
\label{sec:Jarjestelma arkkitehtuuri}

\section{Käyttöliittymä}
\label{sec:Kayttoliittyma}

Käyttöliittymä robotin ojausta varten toteutettiin RaspberryPi -alustalla.
Robotin ohjaus käyttöliittymän kautta tapahtuu kokonaan RPI:n liitetyn ulkoisen USB näppäimistön kautta.

\subsection{USB käyttöliittymälaitteiden tulkitseminen}
\label{sub:USB kayttoliittymalaitteiden tulkitseminen}

Näppäimisän kautta annetut käskyt luetaan erityisestä \verb+/dev/input/eventX+ tiedostosta. 

\section{Ohjaus teoria}
\label{sec:Ohjaus teoria}

Robotin toteuttamista varten sovellettiin useaa eri mateemaattista mallinnusta robotin oletetusta käyttäytymisestä.
Mallit phojautuvat hyvin tunnettuihun ohjausmalleihin joiden oikea toimivuus on näyhty käytännön sovelluksissa jo ennestään.
 

\subsection{Kineettinen malli}
\label{sub:Kineettinen malli}

Mekaanisesti robotti on rakennettu niin, että liikkuminen toteutetaan ensisijaisesti kahdella laitteen sivuilla sijaitsevan renkaan avulla.
Robotti liikkuu käyttäen kahta ns. differentiaalista ajo-moottoria.
Käytännössä tämä tarkoittaa sitä, että sivulla olevia moottoerita voidaan ohjeta toisitaan riippumatta.
Robotti kääntyy kun renkaat pyörivät eri nopeutta suhteessa toisiinsa.
Lisäksi robotilla on vapaasti liikkuvia tukipyöriä jotta se pytyy tasapainossa.

Robotin ohjaamista varten käytetty matemaattinen malli on seuraavanlainen:
\begin{align}
    \label{equ:diff-drive-implementation}
    \begin{dcases}
        \dot x = \cfrac{R}{2} (v_r + v_l) \cos (\phi)\\
        \dot y = \cfrac{R}{2} (v_r + v_l) \sin (\phi)\\
        \dot \phi = \cfrac{R}{L} (v_r - v_l)   
    \end{dcases}
\end{align}

Robotin oletetaan kulkevan tasaisen tason pinnalla. $x$ ja $y$ kuvaavat laitteen paikkaa tasolla, $\phi$ robotin etuosan osittamaa suuntaan, $v_r$ ja $v_l$ ovat oikean sekä vasemmanpuolisen renkaan pyörimisnopeudet. Lisäksi mallissa esiintyy vakio $R$ joka on ohjaukseen käytettyjen renkaiden säde. Vakio $L$ on renkaiden etäisyys toisistaan.


---

Yllä esitelty malli toimii hyvin ohjausta varten.
Ohjausalgorimejä kehittäessä päädyttiin kuitenkin siihen tulokseen, että järjestelmää kannattaa käsitellä  yksinkertaisemalla mallilla. Robotin liikkeitä on hankala hahmottaa ajattelemalla pelkästään renkaiden pyörimisnopeutta.

Ohjausalgoritmien kehittämistä varten käytetty matemaattinen malli on seuraavanlainen:
\begin{align}
    \label{equ:diff-drive-design}
    \begin{dcases}
        \dot x = v \cos(\phi)\\
        \dot y = v \sin(\phi)\\
        \dot \phi = \omega
    \end{dcases}
\end{align}

Mallin avulla voidaan määrittää robotin liikke käyttäen hyväksi pelkästään sen nopeutta $v$ sekä kulmanopeutta $\omega$.

Soveltamalla malleja (\ref{equ:diff-drive-implementation}) ja (\ref{equ:diff-drive-design}) saadaan korrelaatio robotin translaation sekä rotaation ($v$ ja $\omega$) ja renkaiden pyörimisnopeuksien ($v_r$ ja $v_l$) välille:
\begin{align}
    \label{equ:diff-drive-correlation}
    &
    \begin{dcases}
        v &= \cfrac{R}{2} (v_r + v_l)\\
        \omega &= \cfrac{R}{L} (v_r - v_l)\\
    \end{dcases}
    &
    \begin{dcases}
        v_r &= \cfrac{2v + \omega L}{2R}\\
        v_l &= \cfrac{2v - \omega L}{2R}\\
    \end{dcases}
\end{align}



\subsection{Ohjaus silmukan malli}
\label{sub:Ohjaus silmukan malli}

PID -ohjain


\section{Kommunikointi protokolla}
\label{sec:Kommunikointi protokilla}

Robotti vastaanottaa käskyjä, ja lähettää sensoreista luettua tietoa takaisin sarjaväylän yli.

Sarjaväylän langaton tiedonsiirto on toteutettuu 2.4 GHz pakettiradio moduuleilla.
Moduulien tiedosiirtokyky on riittävän luotettava, että 


\section{Oppitulosket}
\label{sec:Oppitulosket}

Projektin tarkoituksena oli syventyä järjestelmäläheiseen C-kilen ohjelmointiin AVR sekä ARM prosessoriarkkitehtuureilla ja yelisemmin kielen eri ominaisuuksiin.

USB käyttöliitymälaitteiden raa'an datan tulkistemista varten ei löytynyt kovin perusteellista oppimateriaalia.
Suuri osa työskently- ja oppimisprosessia koostui eri asiaankuuluvien C otsaketiedostojen lukemisesta, koska tietoa ei suurikaan muualta löytynyt helposti. 

--

Muita kiinnostavia oli esimerkiski keskiä millä datatyypillä AVR:n erikoistoimintorekisterit (SFR) ovat esitetty.
Esitytavan tunteminen oli olennaista jotta pystyi esimerkiksi kirjoittamaan funktioita joille annetaan I/O -nasta argumenttina.

\begin{verbatim}
    volatile uint8_t *
\end{verbatim}

--    

\verb+Struct+ tietorakentiden vertaileminen eri arkkitehtuureilla oli myös kiinnostavaa.
RPI:n 32-bittin ARM arkkitehtuuri pyrkii asettamaan datan muistiosotteisiin jotka ovat neljällä jaollisia. 8-bittinen AVR ei kuitenkaan muistiosoitteien jaollisuudesta välitä.
Tämä johtaa siihen, että structiin pakattua dataa ei voida turvallisesti siirtää suoraan ARM arkkitehtuurilta AVR arkkitehtuurille ilman että erikseen vamistetaan että tietorakenteiden esitys todella on sama molemmilla arkkitehtuuriella.
Kirjoittaessa koodia jota on tarkoitus suorittaa molemilla arkkitehtuureilla pitää pitää mielessä että myös datatyyppien koot eroavat toisistaan. AVR:n \verb+int+ muuttuja on 16-bittinen, kun 32-bittisen ARM:n datatyyppi on 32-bittinen.
Yleispätevää koodia saa kuitenkin kätevästi kirjoitettua käyttämällä \verb+<inttypes.h>+ määrittämiä datapyyppejä kuten \verb+uint8_t+ ja \verb+int32_t+
--
printf puskurointi

\begin{itemize}
    \item POSIX
    \item ioctl
    
\end{itemize}

  
--

Ojelmoitaessa PWM lähtöä moottoreiden ohjaamista varten oli vahingossa ATmega8 datalehti käytössä ATmega328 datalehden sijaan. Mikrokontrollert ovat melkein samanlaiset, mutta niisä ei kuitenkaan ole identtisiä ajastimia. 


\end{document}